%%%%%%%%%%%%%%%%%%%%%%%%%%%%%%%%%%%%%%%%%
% NIH Grant Proposal for the Specific Aims and Research Plan Sections
% LaTeX Template
% Version 1.1 (26/12/19)
%
% This template originates from:
% http://www.LaTeXTemplates.com
%
% Original author:
% Erick Tatro (erickttr@gmail.com) with modifications by:
% Vel (vel@latextemplates.com)
%
% Adapted from:
% J. Hrabe (http://www.magalien.com/public/nih_grants_in_latex.html)
%
% License:
% CC BY-NC-SA 3.0 (http://creativecommons.org/licenses/by-nc-sa/3.0/)
%
%%%%%%%%%%%%%%%%%%%%%%%%%%%%%%%%%%%%%%%%%

%----------------------------------------------------------------------------------------
%	PACKAGES AND OTHER DOCUMENT CONFIGURATIONS
%----------------------------------------------------------------------------------------

\documentclass[11pt, notitlepage]{article} % Default font size and suppress title page

\usepackage[utf8]{inputenc} % Required for inputting international characters
\usepackage[T1]{fontenc} % Output font encoding for international characters
% A note on fonts: As of 2019, NIH allows Arial, Georgia, Helvetica, and Palatino Linotype. Georgia and Arial are commercial fonts so you will need to use XeLaTeX and have them installed on your machine to use them. Palatino & Helvetica are available as free LaTeX packages so select the one you want and comment out the other.
\usepackage{palatino} % Palatino font
\linespread{1.05} % A little extra line spread is better for the Palatino font
%\usepackage{helvet} % Helvetica font
\renewcommand*\familydefault{\sfdefault} % Use the sans serif version of the font

\usepackage{amsfonts, amsmath, amsthm, amssymb} % For math fonts, symbols and environments
\usepackage{graphicx} % Required for including images
\usepackage{booktabs} % Nice rules in tables
\usepackage{wrapfig} % Required for text to wrap around figures and tables
\usepackage[labelfont=bf]{caption} % Make figure numbering in captions bold
\usepackage[top=0.5in,bottom=0.5in,left=0.5in,right=0.5in]{geometry} % Page margins
\pagestyle{empty} % Suppress headers and footers

\hyphenation{ionto-pho-re-tic iso-tro-pic fortran} % Specifies custom hyphenation points for words or words that shouldn't be hyphenated at all


%----------------------------------------------------------------------------------------
\begin{document}

%----------------------------------------------------------------------------------------
%	SPECIFIC AIMS
%----------------------------------------------------------------------------------------

\section*{Specific Aims}

% Following this sort of format: https://www.biosciencewriters.com/NIH-Grant-Applications-The-Anatomy-of-a-Specific-Aims-Page.aspx

% First Paragraph: 
Vision provides crucial information for successfully moving through the environments of daily life.  There is a rich and growing body of literature that describes vision and visual perception in response to dynamic and realistic environments [CITE], as well as the details of the biomechanics of walking in natural environments [CITE].  However, the vast majority of experiments investigating the sensorimotor processes underpinning the visually-guided walking are conducted in isolation, focusing primarily on visual perception or motor function [CITE].  As a result, there is a lack of data to support the development of a normative description of the sensorimotor processes involved in walking.  This significantly hinders the development of models of the cognitive planning and visual information gathering processes that integrate the details of visual processing AND the bimoechanics of human movement. Understanding these basic sensorimotor processes is critical to human health as we age, as there is considerable evidence that visual impairment and other changes associated with aging put individuals at a high risk for falls.

The overarching goal of this proposal is to \textbf{develop an integrated model of the visuomotor processes that support movement through real-world environments}. It will provide a detailed and integrated account of the visual information gathering and cognitive/motor planning processes that support walking.  We will take into consideration the role of divided attention and the way that it shapes the coordination of gaze and gait by limiting visual information gathering and cognitive processing.  This work will be informed by the collection of an \textbf{integrated visuomotor dataset (eye tracking and full-body movement through real-world environments)} and a series of controlled-laboratory experiments with protocols designed to mimic the visually-guided walking observed in the natural environment.  These complementary approaches will enable the observation of real-world behavior, while still providing precise laboratory measurements to test specific hypotheses related to the \textbf{dynamics of visual information-gathering and motor planning}.  

We are uniquely positioned tackle this set of scientific questions, having developed both environment- and laboratory-based data collection techniques that produce integrated visuomotor datasets for full-body movement.  The fields of vision science, neuroscience, and biomechanics are at a critical junction as advances in machine learning increase the capacity for processing and analyzing big multi-modal data.  However, the success of such efforts is dependent on the content and quality of the data that exist.  Our proposed work will result in a high-quality, open-source, visuomotor dataset.  Furthermore, each aim includes planned technical deliverables.  These open-source solutions will lower the barrier for creating integrated visuomotor datasets. Because of the rarity of such datasets for full-body movement and the current difficulty of producing them, these technical deliverables are a central contribution of the proposal, but one that is deeply intertwined with the scientific goals.


\begin{description}
	\item[SA1: Information gathering and motor planning during full-body movement through real-world environments.]{There is a lack of normative baseline data on how individuals use their vision to actively select the information that guides walking through complex environments. Our approach will be to collect an integrated visuomotor dataset (including body movements, eye movements and the environment) in 50 adults with typical vision and motor function. We will analyze motor planning strategies, assess how the moment-to-moment instability of gait impacts gaze behavior, and identify adaptive gaze patterns. More generally, we will model the coordination of gaze/gait during visually-guided walking, establishing typical gaze patterns in the context of the motor behavior. \textit{SA1} will provide a comprehensive description of the sensorimotor processes that underlie visually-guided walking. \textit{Technical Deliverables (SA1):} comprehensive visuomotor dataset; open-source documentation of the hardware infrastructure for collecting low-cost, high-quality integrated visuomotor data; open-source software processing pipeline for integrating multi-modal dataset into the same spatial reference frame.}

 
% \begin{description}
% 	\item[SA1: Laser Skeletons]{Characterize the coordination of gaze and gait during walking in complex terrains.}
% 	\item[Rationale:] {There is a lack of normative baseline data on how individuals actively use their visual systems to select information for guiding gross sensory motor movements, such as walking, in complex environments. Establishing a better understanding of gaze patterns and motor planning during walking will provide valuable insights into visual processing and movement coordination.}
% \item[Approach:]{We will collect data from 50 adults with typical vision to comprehensively describe their eye movements during walking in various terrains. We will analyze motor planning strategies, changes in gaze behavior during instability, and identify advantageous gaze patterns during moments of instability.}
% \item[Technical Deliverables:]{ A comprehensive dataset of eye movement patterns in adults. Detailed analyses of motor planning strategies and gaze behavior during walking.  Data/Analysis infrastructure that support the re-use of the dataset by other groups. Open source data collection infrastructure that supports the collection of new datasets by other groups.}
% \end{description}

    \item[SA2: Testing the spatial and temporal dynamics of visual information gathering and motor planning.] {We have developed a body- and gaze-contingent augmented reality ground plane (3m x 10m) for the presentation of arbitrary 2D walking paths. Preliminary results show that the manipulation of foothold sparsity results in modulation of gaze/gait behavior that mimics the changes due to different terrain complexity observed in natural environments. We will manipulate the availability of visual information based on current body position and gaze location to identify the role of temporal dynamics and peripheral processing in the visual information gathering that supports walking.  \textit{SA2} will identify when and where in the visual field the critical information for foothold selection during walking occurs. \textit{Technical Deliverables (SA2):} open source hardware specifications for "augmented reality ground-plane"; open-source software processing pipeline for integrating laboratory-based multi-modal data into the same spatial reference frame}

    % saving this for later...
    % \textit{Technical deliverables:} detailed documentation of the hardware infrastructure and laboratory protocols the augmented reality ground-plane; integration of the laboratory-based data collection techniques with the software processing pipeline described in SA1. }

%     \begin{description}
% 	\item[Aim 2: Augmented Reality Ground Plane] {Develop laboratory protocols using an augmented reality ground plane to disentangle motor planning strategies and visual information processing strategies observed during walking in natural environments.  }

% \item[]{Rationale: Observational data from natural environments informs this set of laboratory experiments, allowing for controlled examination of visual information and its influence on motor planning and gaze-gait coordination. This approach will enable a deeper understanding of the underlying mechanisms and strategies involved in walking.}
% \item[]{ Approach: We will use an augmented reality ground plane to tightly control what visual information is available and when it is available. By controlling the available visual information, we will test hypotheses derived from the observational data regarding motor planning and gaze-gait coordination.}
% \item[] {Technical Deliverables: A set of laboratory protocols utilizing augmented reality technology, designed to systematically control and manipulate visual information during walking experiments. These protocols will enable researchers to investigate the interaction between visual processing and motor planning in a controlled setting.}
% \end{description}

    \item[SA3: Impact of Divided on Attention on the visuomotor control of walking.] {The data collection effort described in SA1 will include a divided attention condition. Participants will be asked to walk while talking to an experimenter and/or completing tasks on their phone.  Prior work across gait and postural control studies demonstrates that there is a cost to divided attention [CITE]. We will measure the impact of divided attention on the coordination of gaze/gait during full-body movement through real-world environments. \textit{SA3} will provide insight into how visual information gathering and motor planning resources are allocated when individuals are simultaneously engaged in two tasks. \textit{Technical Deliverables (SA3): divided attention extension to comprehensive visuomotor dataset (see SA1)}}.

% \begin{description}
% 	\item[Aim 3: Divided Attention] {Examine the impact of divided attention on the coordination of gaze and gait.}
 
% 	\item{Rationale: In everyday life, people often engage in multiple activities simultaneously while navigating their environment, making it important to understand how divided attention affects gaze and gait coordination. Furthermore, age-related changes in multitasking abilities may impact gaze-gait coordination and warrant investigation.}
 
%     \item{Approach: We will simultaneously collect this data in Aim 1, with participants walking across complex terrains while engaging in divided attention tasks such as playing on their phone or talking to a friend. We will observe and analyze gaze patterns and gait coordination under these divided attention conditions.}
    
%     \item{Technical Deliverables: A comprehensive dataset on gaze and gait coordination during divided attention tasks, providing insights into the impact of multitasking on walking performance and the safe navigation of complex environments.}


\end{description}

% You could potentially end with a summary if you feel like it.
% The expected outcome of the proposed research is a characterization of ... The health relevance of the research resides in understanding ..

%----------------------------------------------------------------------------------------
%	SIGNIFICANCE
%----------------------------------------------------------------------------------------
Walking through complex natural environments requires the robust
integration of our visual and motor systems. The visual and motor
components of visually-guided walking, visual search and the
biomechanics of bipedal locomotion, have been studied extensively, but
largely in isolation.

\begin{itemize}
\item
  Recent work (that we have been heavily involved in) takes important
  first steps to integrated understanding of the visuomotor control of
  walking.
\end{itemize}

Visual search

Gap 1. how visual processing supports human movement in the real-world

Testing in real life. the sequence of events that happen in the real
world.

The large datasets that are being generated only have eye movements and
world content: ego 4d and other long list of stuff. An old idea in
cognitive science and visual perception that has gained traction in the
ML community via reinforcement learning approaches is that the ability
to take action and adjust your own input influences learning. We need
datasets with stored actions to make this happen\ldots{}

Gap 2. The tech to actually study this doesn't exist. We're uniquely
positioned to study it.

New technology is a gap -- it's still a lot of work. We're going

\begin{itemize}
\item
  Established the general patterns of visuomotor control, but now we
  need to:

  \begin{itemize}
  \item
    1. Leverage improved processes for collecting integrated visuomotor
    datasets to collect more precise, expanded datasets.
  \item
    2. Develop laboratory protocols that allow for the
  \end{itemize}
\end{itemize}
that allow for the

%----------------------------------------------------------------------------------------
%	INNOVATION
%----------------------------------------------------------------------------------------

\section*{B. Innovation}

\begin{description}
	\item[B.1. Instructions.]{}
\end{description}

Explain how the application challenges and seeks to shift current research or clinical practice paradigms.

Describe any novel theoretical concepts, approaches or methodologies, instrumentation or interventions to be developed or used, and any advantage over existing methodologies, instrumentation, or interventions.

Explain any refinements, improvements, or new applications of theoretical concepts, approaches or methodologies, instrumentation, or interventions.


%----------------------------------------------------------------------------------------
%	APPROACH
%----------------------------------------------------------------------------------------
\input{approach}

%----------------------------------------------------------------------------------------
%	PROGRESS REPORT
%----------------------------------------------------------------------------------------

\newpage

\section*{5. Progress Report Publication List (Renewal Applications Only)}

List the titles and complete references to all appropriate publications, manuscripts accepted for publication, patents, and other printed materials that have resulted from the project since it was last reviewed competitively. When citing articles that fall under the Public Access Policy, were authored or co-authored by the applicant and arose from NIH support, provide the NIH Manuscript Submission reference number (e.g., NIHMS97531) or the Pubmed Central (PMC) reference number (e.g., PMCID234567) for each article. If the PMCID is not yet available because the Journal submits articles directly to PMC on behalf of their authors, indicate "PMC Journal -- In Process." A list of these journals is posted at: http://publicaccess.nih.gov/submit\_process\_journals.htm.

Citations that are not covered by the Public Access Policy, but are publicly available in a free, online format may include URLs or PMCID numbers along with the full reference (note that copies of these publications are not accepted as appendix material, see Part I Section 5.5.15 for more information).

%----------------------------------------------------------------------------------------
%	PROTECTION OF HUMAN SUBJECTS
%----------------------------------------------------------------------------------------

\newpage

\section*{6. Protection of Human Subjects}

Refer to Part II, Supplemental Instructions for Preparing the Human Subjects Section of the Research Plan.

This section is required for applicants answering "yes" to the question "Are human subjects involved?" on the R\&R Other Project Information form. If the answer is "No" to the question but the proposed research involves human specimens and/or data from subjects applicants must provide a justification in this section for the claim that no human subjects are involved.

Do not use the protection of human subjects section to circumvent the page limits of the Research Strategy.

%----------------------------------------------------------------------------------------
%	INCLUSION OF WOMEN AND MINORITIES
%----------------------------------------------------------------------------------------

\newpage

\section*{7. Inclusion of Women and Minorities}

Refer to Part II, Supplemental Instructions for Preparing the Human Subjects Section of the Research Plan. This section is required for applicants answering "yes" to the question "Are human subjects involved?" on the R\&R Other Project Information form and the research does not fall under Exemption 4.

%----------------------------------------------------------------------------------------
%	INCLUSION OF CHILDREN
%----------------------------------------------------------------------------------------

\newpage

%\section*{8. Targeted/Planned Enrollment} - form to fill out 
\section*{9. Inclusion of Children}

Refer to Supplemental Instructions for Preparing the Human Subjects Section of the Research Plan, Sections 4.4 and 5.7. For applicants answering "Yes" to the question "Are human subjects involved" on the R\&R Other Project Information Form and the research does not fall under Section 4, this section is required.

%----------------------------------------------------------------------------------------
%	VERTEBRATE ANIMALS
%----------------------------------------------------------------------------------------

\newpage

\section*{10. Vertebrate Animals}

If Vertebrate Animals are involved in the project, address each of the five points below. This section should be a concise, complete description of the animals and proposed procedures. While additional details may be included in the Research Strategy, the responses to the five required points below must be cohesive and include sufficient detail to allow evaluation by peer reviewers and NIH staff. If all or part of the proposed research involving vertebrate animals will take place at alternate sites (such as project/performance or collaborating site(s)), identify those sites and describe the activities at those locations. Although no specific page limitation applies to this section of the application, be succinct. Failure to address the following five points will result in the application being designated as incomplete and will be grounds for the PHS to defer the application from the peer review round. Alternatively, the application’s impact/priority score may be negatively affected.

If the involvement of animals is indefinite, provide an explanation and indicate when it is anticipated that animals will be used. If an award is made, prior to the involvement of animals the grantee must submit to the NIH awarding office detailed information as required in points 1-5 above and verification of IACUC approval. If the grantee does not have an Animal Welfare Assurance then an appropriate Assurance will be required (See Part III, Section 2.2 Vertebrate Animals for more information).
The five points are as follows:

\begin{enumerate}
	\item Provide a detailed description of the proposed use of the animals in the work outlined in the Research Strategy section. Identify the species, strains, ages, sex, and numbers of animals to be used in the proposed work.
	\item Justify the use of animals, the choice of species, and the numbers to be used. If animals are in short supply, costly, or to be used in large numbers, provide an additional rationale for their selection and numbers.
	\item Provide information on the veterinary care of the animals involved.
	\item Describe the procedures for ensuring that discomfort, distress, pain, and injury will be limited to that which is unavoidable in the conduct of scientifically sound research. Describe the use of analgesic, anesthetic, and tranquilizing drugs and/or comfortable restraining devices, where appropriate, to minimize discomfort, distress, pain, and injury.
	\item Describe any method of euthanasia to be used and the reasons for its selection. State whether this method is consistent with the recommendations of the American Veterinary Medical Association (AVMA) Guidelines on Euthanasia. If not, include a scientific justification for not following the recommendations.
\end{enumerate}

Do not use the vertebrate animal section to circumvent the page limits of the Research Strategy.

%----------------------------------------------------------------------------------------
%	SELECT AGENT RESEARCH
%----------------------------------------------------------------------------------------

\newpage

\section*{11. Select Agent Research}

Select Agents are hazardous biological agents and toxins that have been identified by DHHS or USDA as having the potential to pose a severe threat to public health and safety, to animal and plant health, or to animal and plant products. CDC maintains a list of these agents. See http://www.cdc.gov/od/sap/docs/salist.pdf.

%----------------------------------------------------------------------------------------
%	MULTIPLE PD/PI LEADERSHIP PLAN
%----------------------------------------------------------------------------------------

\newpage

\section*{12. Multiple PD/PI Leadership Plan}

For applications designating multiple PD/PIs, a leadership plan must be included. A rationale for choosing a multiple PD/PI approach should be described. The governance and organizational structure of the leadership team and the research project should be described, including communication plans, process for making decisions on scientific direction, and procedures for resolving conflicts. The roles and administrative, technical, and scientific responsibilities for the project or program should be delineated for the PD/PIs and other collaborators.

If budget allocation is planned, the distribution of resources to specific components of the project or the individual PD/PIs should be delineated in the Leadership Plan. In the event of an award, the requested allocations may be reflected in a footnote on the Notice of Grant Award.

%----------------------------------------------------------------------------------------
%	CONSORTIUM/CONTRACTUAL ARRANGEMENTS
%----------------------------------------------------------------------------------------

\newpage

\section*{13. Consortium/Contractual Arrangements}

Explain the programmatic, fiscal, and administrative arrangements to be made between the applicant organization and the consortium organization(s). If consortium/contractual activities represent a significant portion of the overall project, explain why the applicant organization, rather than the ultimate performer of the activities, should be the grantee. The signature of the Authorized Organization Representative on the SF424 (R\&R) cover component (Item 17) signifies that the applicant and all proposed consortium participants understand and agree to the following statement:

\emph{The appropriate programmatic and administrative personnel of each organization involved in this grant application are aware of the agency's consortium agreement policy and are prepared to establish the necessary inter-organizational agreement(s) consistent with that policy.}

%----------------------------------------------------------------------------------------
%	RESOURCE SHARING
%----------------------------------------------------------------------------------------

\newpage

%   \section*{14. Letters of Support} - letters to attach
\section*{15. Resource Sharing}

NIH considers the sharing of unique research resources developed through NIH-sponsored research an important means to enhance the value and further the advancement of the research. When resources have been developed with NIH funds and the associated research findings published or provided to NIH, it is important that they be made readily available for research purposes to qualified individuals within the scientific community. See Part III, 1.5 Sharing Research Resources.

\begin{enumerate}
	\item{Data Sharing Plan:} Investigators seeking \$500,000 or more in direct costs (exclusive of consortium F\&A) in any year are expected to include a brief 1-paragraph description of how final research data will be shared, or explain why data-sharing is not possible. Specific Funding Opportunity Announcements may require that all applications include this information regardless of the dollar level. Applicants are encouraged to read the specific opportunity carefully and discuss their data-sharing plan with their program contact at the time they negotiate an agreement with the Institute/Center (IC) staff to accept assignment of their application. See Data-Sharing Policy or http://grants.nih.gov/grants/guide/notice- files/NOT-OD-03-032.html.
	\item{Sharing Model Organisms:} Regardless of the amount requested, all applications where the development of model organisms is anticipated are expected to include a description of a specific plan for sharing and distributing unique model organisms or state why such sharing is restricted or not possible. See Sharing Model Organisms Policy, and NIH Guide NOT-OD-04-042.
	\item{Genome Wide Association Studies (GWAS):} Applicants seeking funding for a genome-wide association study are expected to provide a plan for submission of GWAS data to the NIH-designated GWAS data repository, or an appropriate explanation why submission to the repository is not possible. GWAS is defined as any study of genetic variation across the entire genome that is designed to identify genetic associations with observable traits (such as blood pressure or weight) or the presence or absence of a disease or condition. For further information see Policy for Sharing of Data Obtained in NIH Supported or Conducted Genome-Wide Association Studies, NIH Guide NOT-OD-07-088, and http://grants.nih.gov/grants/gwas/.
\end{enumerate}

%----------------------------------------------------------------------------------------
%	BIBLIOGRAPHY
%----------------------------------------------------------------------------------------

\newpage

\bibliography{NIHGrant} % Use the NIHGrant.bib file for the reference list, replace with your own
\bibliographystyle{nihunsrt} % Use the custom nihunsrt bibliography style included with the template

%----------------------------------------------------------------------------------------

\end{document}