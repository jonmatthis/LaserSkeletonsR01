
\section*{Specific Aims}

% Following this sort of format: https://www.biosciencewriters.com/NIH-Grant-Applications-The-Anatomy-of-a-Specific-Aims-Page.aspx

% First Paragraph: 
Vision provides crucial information for successfully moving through the environments of daily life.  There is a rich and growing body of literature that describes vision and visual perception in response to dynamic and realistic environments [CITE], as well as the details of the biomechanics of walking in natural environments [CITE].  However, the vast majority of experiments investigating the sensorimotor processes underpinning the visually-guided walking are conducted in isolation, focusing primarily on visual perception or motor function [CITE].  As a result, there is a lack of data to support the development of a normative description of the sensorimotor processes involved in walking.  This significantly hinders the development of models of the cognitive planning and visual information gathering processes that integrate the details of visual processing AND the bimoechanics of human movement. Understanding these basic sensorimotor processes is critical to human health as we age, as there is considerable evidence that visual impairment and other changes associated with aging put individuals at a high risk for falls.

The overarching goal of this proposal is to \textbf{develop an integrated model of the visuomotor processes that support movement through real-world environments}. It will provide a detailed and integrated account of the visual information gathering and cognitive/motor planning processes that support walking.  We will take into consideration the role of divided attention and the way that it shapes the coordination of gaze and gait by limiting visual information gathering and cognitive processing.  This work will be informed by the collection of an \textbf{integrated visuomotor dataset (eye tracking and full-body movement through real-world environments)} and a series of controlled-laboratory experiments with protocols designed to mimic the visually-guided walking observed in the natural environment.  These complementary approaches will enable the observation of real-world behavior, while still providing precise laboratory measurements to test specific hypotheses related to the \textbf{dynamics of visual information-gathering and motor planning}.  

We are uniquely positioned tackle this set of scientific questions, having developed both environment- and laboratory-based data collection techniques that produce integrated visuomotor datasets for full-body movement.  The fields of vision science, neuroscience, and biomechanics are at a critical junction as advances in machine learning increase the capacity for processing and analyzing big multi-modal data.  However, the success of such efforts is dependent on the content and quality of the data that exist.  Our proposed work will result in a high-quality, open-source, visuomotor dataset.  Furthermore, each aim includes planned technical deliverables.  These open-source solutions will lower the barrier for creating integrated visuomotor datasets. Because of the rarity of such datasets for full-body movement and the current difficulty of producing them, these technical deliverables are a central contribution of the proposal, but one that is deeply intertwined with the scientific goals.





------

\section*{ FROM GOOGLE DOC}

When humans walk through the world, their eye movements and body movements are precisely coordinated enabling impressive performance in complicated environments. Past studies have generally been small, and either focused on vision or on the biomechanics of walking, with relatively little crosstalk. A major impediment for the combination is a lack of good datasets, we would want to know body movements, eye movements, and the environment, all at the same time. Such datasets also hold back the development of strong predictive models, as neither eye movements nor body movements can meaningfully be separated during real-world walking. It also holds back research to ask to which level eye-movements and body movements can be seen as jointly optimized. What is missing is a big meaningful dataset, along with the relevant descriptive and normative modeling.

Multiple developments in movement science and neural networks make addressing these problems now realistic. Eye and body movements can now be efficiently traced using pose tracking technologies that I learned to use during my Postdoc. Overall movement of eyes and body can now be modeled with neural networks, that I have used since my PhD thesis. And asking questions about optimality of movements has been the core of my postdoctoral work. The combination of these skills now promises to allow us to move the joint study of walking and eye movement from the laboratory into the real world.


\section*{Kate writes 3 questions and 3 paragraphs to describe what might be 3 aims}

\begin{description}
    
      \item[Q1: What gaze strategies do humans use for foothold finding?]{This is a real-world visual search task.  We know from our preliminary data that search is driven to a large degree by biomechanics.  But we have more questions:  What are the sequences of eye movements that people make? What are the timing of eye movements relative to steps (I've been looking at data... I think there's something here)?  What is the impact of divided attention -- Does task switching reveal something? What is the necessary information?  When there is not other task they can do whatever they want.
    
    Notes from conversation with Jon:  ARGP will help us say something about fixations to footholds and distractors. To really deconstruct the visual search.  Separate fixations into where we stepped and where we looked and distractors/footholds.  
    
    Fixations for information gathering vs. steering and posture.}
    
    \item[Q2: What makes a good foothold?]{Foothold locations are determined by a variety of factors: getting to a location, minimizing the energetic costs of walking, avoiding paths that change in height or direction, stability and/or “flatness” of the ground. Basically, you want to get from point A to B, but without getting too tired or stepping on wobbly rocks. How do walkers trade off these different costs?

    A number of biomechanical definitions of what makes a good foothold (Jon... list please).  How can people identify those with their visual systems?
    }

  
    
    \item[Q3: How do humans adjust gait to accommodate upcoming obstacles and changes in direction?]{a. Let's fit a walker and measure step efficiency and how it changes. b. Let's repeat Kuo's obstacle adjustment experiment for lateral movements (if I knew what an ARGP was... I'd say we should use that). c. Let's take Kuo's strategy and apply it to outdoor data.}

\end{description}