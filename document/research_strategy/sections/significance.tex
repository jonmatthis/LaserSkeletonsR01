
\textbf{A.1 Human behavior in real-world environments relies on sequences of sensorimotor decisions}

Visual perception and human movement 

- ocular motor control 
- reaching





understanding things in the "real world" requires integrated measurement/understanding.  Visual perception, movement, other things simultaneously


An old idea in cognitive science and visual perception that has gained traction in the ML community via reinforcement learning approaches is that the ability to take action and adjust your own input influences learning. We need datasets with stored actions to make this happen

this is not studied.  we don't have datasets to take advantage of this.

- We don't have a model of human walkers. And frankly we don't even have the right dataset.  Most datasets aren't egocentric.  The datasets that are do not include complete body pose information that allows us to understand the relationship between visual perception and action.

The large datasets that are being generated only have eye movements and world content: ego 4d and other long list of stuff. 

attempts by the vision science community to gather gaze data from outdoor environments (Pelz \& Rothkopf, 2007; t'Hart \& Einhauser, 2012; Fulsham et al, 2011) Sprague et al (2015) and Cooper et al (2016) Vasha Dutell \& Agostino Gibaldi, laser skeletons, 

attempts by the walking community to gather walking data from outdoor environment: are there really any of these for full body? adamczyk, JON

volume is small, not repeatable, 

the interdisciplinary nature of the problem is part of the challenge for creating the tools.



\textbf{A.2 Humans have a robust visuomotor control system that supports walking in complex environments.}

Closed loop system:

[stepping influences seeing] Biomechanics of walking constrains visual behavior 
- we look 2-5 steps ahead (Jon lab stuff, patla stuff, Jon/Mary/Kate stuff)

[seeing influences stepping] 
Kuo: incorporate upcoming obstacles to maximize energetic efficiency

An old idea in cognitive science and visual perception that has gained traction in the ML community via reinforcement learning approaches is that the ability to take action and adjust your own input influences learning. We need datasets with stored actions to make this happen

\noindent \textbf{A.3 Visual search patterns are efficient and task-dependent}

\noindent There is strong evidence from investigations into visual search for targets within natural scene statistics that search patterns are highly fixation-efficient and subsequent eye movements are made to maximize the probability of finding the target \cite{najemnik_optimal_2005}. Additionally, there is a body of research which demonstrates that visual search is highly task-dependent \cite{jovancevic-misic2009, tong2017, zhang2018, hayhoe2005, tatler2011, rothkopf2016}. This research demonstrates that the visual system works to provide information relevant to the goals of the perceiver, such that eye movements are efficiently made to serve the current task demands and reduce uncertainty \cite{Matthis2018}. 

EXAMPLE FROM MORE TYPICAL VISUAL SEARCH LITERATURE

What information is most relevant for the task of locomotion?

Recently, we conducted the first investigation into eye movements made while traversing natural outdoor terrains of varying difficulty \cite{Matthis2018,hayhoe2018} (Fig. 1). Examination of this data demonstrates that eye movements made during visual search are likely influenced by the biomechanical constraints of the locomotor system: walkers fixate where they want to step as opposed to visually salient terrain. Critically, this observation – that the proprioceptive information of biomechanics might influence visual search – is substantiated by recent work in biomechanics.

Examinations of bipedal biomechanics consistently demonstrate that human locomotion is energetically efficient\cite{Kuo2002,Donelan2002}. Recently, researchers have shown that the humans are constantly adapting their gait to maintain energetically optimal movements even seconds after perturbation \cite{selinger2015}. This, combined with evidence that humans can readily perceive energetically efficient\cite{warren1984} or optimal body movements \cite{weast-knapp2019}, suggests that the visual system must be seeking information which allows for energetically efficient foot placements.

ANOTHER PARAGRAPH HERE -- So where does this leave us?


\textbf{A.4 Something about energetic efficiency}

PNAS paper intro had a big schtick about this.  
we can take into account our biomechanics when we walk around.

humans are good at moving and adjusting movement in a way that doesn't interfere much with being a large lumpy biped.

check out the PNAS and the K99

\textbf{A.5 Summary of significance.} 

overall scientific premise of this proposal is bolstered by past research showing that...

This proposal is significant because it is designed to ...


%  Insomnia is an under-appreciated, yet highly prevalent, comorbidity for PLWH with the possibility of far-reaching health consequences. The overall scientific premise of this proposal is bolstered by past research showing that insomnia is a potent driver of pain and poor physical function, with inflammation as a key biologic mechanism linking insomnia with each. This is particularly relevant for PLWH given the alarmingly high rate of developing a chronic pain condition as they age. Although yet to be directly tested in PLWH, it stands to reason that inflammation might link insomnia and pain in PLWH given specific characteristics of HIV (i.e., envelope glycoproteins) previously shown in animal studies to underlie all three factors. This proposal is significant because it is designed to specifically address whether insomnia promotes pain, inflammation, and poor physical functioning in PLWH 


  Furthermore, this gap restricts our understanding of how the visuo-locomotor system might be affected by aging and disease (e.g., retinal CITE, motor CITE).  



\begin{figure}[h]
\centering
\includegraphics[width=\textwidth]{document/figures/Figure-1-outdoor-walking.pdf}
\caption{\textbf{Outdoor measurements of the visuo-locomotor system.} A. ?? B. ??}
\end{figure}
