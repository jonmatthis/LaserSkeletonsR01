
NOT CHANGED YET

\textbf{\underline{Rationale:}}  The main goal of this aim is to is to ....

We will examine the specific fixation strategies used by walkers during foothold finding.  We will evaluate how these fixation strategies are influenced by the biomechanical constraints of walking. What visual and biomechanical information is most predictive of the location of the next footfall and/or the next gaze location?  How is that information extracted from the retinocentric reference frame of the walker?


\textbf{\underline{Aim 1a: Gaze strategies in natural terrain}}



\noindent\underline{\textbf{Research Design:}} The resulting data will provide the opportunity to answer a variety questions about visual perception, gaze, gait, the environment, and the coordination of these during walking.

i. \emph{How do walkers allocate gaze (spatially and temporally) during foothold finding? What is the planning horizon?}

Previous work has established that people allocate their gaze to locations 2-5 footholds ahead during foothold finding [CITE], shown also in Figure XXx. These findings did not have the precision afforded by the photogrammetry pipeline described above. Preliminary analyses confirm previous conclusions that walkers are allocating gaze to upcoming footholds, as the increased precision leads to the emergence of ``hills of gaze'',
Figure XXx\ldots{} For each individual we will quantify the allocation of gaze to upcoming footholds, establishing the variability in gaze allocation across a larger population of individuals. The increased precision also allows the classification of gaze locations by their
foothold, or as a non-foothold search location, further distinguishing between non-foothold search locations near the path and those that are part of search sequences before turns (i.e., paths not taken; see Figure XXx [sequence of gaze locations, classified]). For each individual we will quantify the \% time they spend on search locations that become
footholds, the \% time they spend on search locations that don't become footholds and how often those are part of `paths-not-taken'. As participants walk each path twice, we will measure the variability in gaze allocation both with-in and across participants to establish measurements of variability in gaze allocation for individuals with typical visual and motor systems. Establishing the individual variability for those with typical vision is important as previous work demonstrates that an impairment to visual processing leads to the allocation of gaze closer to the body [CITE binocular walking paper]. Thus, changes in gaze allocation may serve as a behavioral biomarker of issues in visual processing.

Relying on the specified gaze locations (e.g., foothold 1, foothold 2, \ldots, non-foothold location, path-not-taken location), we will quantify/model the gaze sequences present during walking in complex terrains. Such sequences can easily be summarized as a markov chain, quantifying the transition probability from one gaze location to the next. We expect that this analysis will capture the broad structure of eye movements in this task. However, we also recognize that the choice of the next gaze location is likely dependent on more than just the current gaze location, a known limitation of this type of Markovian analysis and there are likely common sequences (gaze strategies) that are not captured. Recent work in task-based eye movement analysis has had success in identifying eye movement strategies using Hidden Markov Models (HMMs) and predicting intent using Recurrent Neural Networks (RNNs, specifically LSTM). We will use these modeling approaches to capture the additional complexity in the gaze sequences of walkers.

The larger size of the dataset will offer a unique opportunity to study the sequences of searches on `paths-not-taken' and examine sensorimotor decision-making in the context of path planning. As these events are less common on any individual walk, access to a large dataset is necessary. Preliminary analyses suggest that [FIGURE OUT HOW TO SAY THIS PROPERLY] humans explore their path options in stereotyped ways\ldots{} blah blah blah

ii. \emph{How is gaze allocation modulated by step-to-step gait
efficiency?}

From previous work, we know that terrain difficulty impacts gaze [CITE]. However, many factors change across terrains. In the more difficult terrains, one of central changes to gait is a greater deviation from the preferred gait (i.e., decreased gait efficiency). This accommodates the fewer available footholds due to increased terrain complexity. Here we look to isolate within a terrain the impact of the current level of stability on gaze allocation.

All people have a preferred gait and we can measure each person's preferred gait parameters (e.g., ...) as they walk on flat terrain (pavement). Then we can measure the step-to-step efficiency as the inverted deviation from that preferred gait (see [CITE] for example).  --> (using GLM approaches, with history kernels) relate that deviation to look-ahead distance, look-ahead time and gaze location. We expect that individuals will modulate their gaze to do longer term path planning and foothold finding during periods of instability, i.e. their gaze will be farther down the path during periods of stability.

vi. [COMMENTS ON MODELING? That point to the end of Aim II]

\noindent \underline{\textbf{Experimental concerns.}}

\begin{itemize}
\item
  This seems hard → Validated in previous published studies that we were
  authors in.
\item
  What if there's learning? → We will complete the walking bouts in a
  randomized order twice. Participants will walk the path for the
  experiment once before. We don't expect measurable effects of learning
  on this time scale but we will be able to check. If the first 10
  participants show evidence of learning effects across sessions we can
  consider collecting data at two sites.
\end{itemize}


\textbf{Aim 1b: }
