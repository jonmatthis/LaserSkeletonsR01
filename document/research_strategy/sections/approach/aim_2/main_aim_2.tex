AUGMENTED REALITY GROUND PLANE - Laboratory-precision measurements of
visually-guided walking

\emph{Rationale:}

Why do we need Laboratory-precision measurements?

\emph{General Methods:} The Augmented Reality Ground-Plane (ARGP,
developed at Northeastern) is a projector-based, 14m-long indoor walking
path. The content of the groundplane is displayed using a series of X
projectors and is body-contingent so it can be updated as a function of
the current position of the participant.

For each of the experiments described below, we will recruit groups of
participants (n\textgreater=10) from the Northeastern Community.

Preliminary data demonstrates that the manipulation of the foothold
density (\# of available footholds) broadly mimics the changes observed
across terrains of different difficulty. As in the outdoor data, walkers
fixate the footholds in the upcoming terrain. Like rough terrain, more
difficult ARGP walking trials with few available footholds result in
gaze allocation closer to the body.

Manipulations: foothold density, path tortuosity, paths with variable
foothold density

(by decreasing density in one part of the path we can create path
situations.)

Double-pass experiments

\emph{Research Design:}

\emph{i.}

\begin{enumerate}
\def\labelenumi{\Alph{enumi}.}
\item
  Are speed fluctuations and deviations from preferred gait repeatable
  with-in and across participants dependent on the path content? Are
  gaze paths repeatable within and across participants?
\item
  What is the planning horizon? Measured as in Darici \& Kuo 2023.
  Measured by gaze paths as well, keeping in mind that we might use
  central and peripheral vision. Look at where footholds land in the
  visual field.
\item
  Do people exhibit optimal control when we take into account the
  changes to the path characteristics (path shape and foothold density)?
  {[}Is there previous work to suggest that lateral deviations in path
  should cause the kinds of changes that the vertical blocks cause in
  Darici \& Kuo 2023? Or do we have to establish this? Jon says
  yes\ldots{} if you have to take a shorter step or move more laterally
  you should adjust (read his PNAS paper:
  \href{https://www.pnas.org/doi/epdf/10.1073/pnas.1611699114}{\ul{https://www.pnas.org/doi/epdf/10.1073/pnas.1611699114}}
\item
  Varying ``terrain'' step density over the course of a single path to
  force deviations in efficiency
\item
  Divided Attention -- it's all gonna change friend
\end{enumerate}

\#\#\#\#\# END OF GRANT NOW (we're dumping the separate models section)

MODELS

These should account for laboratory and world data

\begin{enumerate}
\def\labelenumi{\Alph{enumi}.}
\item
  Inverted optimal control model of walking. Apply Darici and Kuo 2023
  to our outdoor data. We know the environment, height changes, overall
  terrain, possible foothold locations.... Can we predict the deviations
  for the idealized pavement walker with an inverse optimal control
  model that takes into account the environment (height changes,
  possible foothold locations)?

  \begin{enumerate}
  \def\labelenumii{\alph{enumii}.}
  \item
    Combining optimal control model developed in Aim II and the one in
    Darici \& Kuo.
  \item
  \end{enumerate}
\item
  GPT
\item
  RNNs to predict next x visual frames, next footsteps, next gaze
  locations, next head movements

  \begin{enumerate}
  \def\labelenumii{\alph{enumii}.}
  \tightlist
  \item
  \end{enumerate}
\end{enumerate}

Predict saccades (or footholds) from world-reference frame information

Predict saccades (or footholds) from retino-centric information

Use to synthesize artificial terrains to understand how these different
sources of information work and what tradeoffs there are.

\end{document}
